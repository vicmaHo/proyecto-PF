\documentclass[conference]{IEEEtran}
% \IEEEoverridecommandlockouts
% The preceding line is only needed to identify funding in the first footnote. If that is unneeded, please comment it out.
\usepackage{tabularx}
\usepackage{booktabs}
\usepackage{cite}
\usepackage{amsmath,amssymb,amsfonts}
\usepackage{algorithmic}
\usepackage{graphicx}
\usepackage{textcomp}
\usepackage{xcolor}
\def\BibTeX{{\rm B\kern-.05em{\sc i\kern-.025em b}\kern-.08em
    T\kern-.1667em\lower.7ex\hbox{E}\kern-.125emX}}
\usepackage{listings}
\usepackage{color}
\usepackage{booktabs}  % Para líneas de alta calidad en las tablas
\definecolor{dkgreen}{rgb}{0,0.6,0}
\definecolor{gray}{rgb}{0.5,0.5,0.5}
\definecolor{mauve}{rgb}{0.58,0,0.82}
\usepackage{hyperref}
\hypersetup{
    colorlinks=true,    % Activa la coloración de enlaces
    linkcolor=black,    % Color de enlaces internos (por ejemplo, índice)
    urlcolor=blue,      % Color de enlaces a URLs
    citecolor=red       % Color de enlaces de citas bibliográficas
}

% \usepackage{geometry}
% \usepackage{graphicx}
% \geometry{letterpaper, top = 2.5cm, right = 2.5cm, left = 2.5cm, bottom = 2.5cm}


\lstdefinelanguage{scala}{
  morekeywords={abstract,case,catch,class,def,%
    else,extends,false,final,finally,%
    for,if,implicit,import,match,mixin,%
    new,null,object,override,package,%
    private,protected,requires,return,sealed,%
    super,this,throw,trait,true,try,%
    type,val,var,while,with,yield},
  otherkeywords={=>,<-,<\%,<:,>:,\#,@},
  sensitive=true,
  morecomment=[l]{//},
  morecomment=[n]{/*}{*/},
  morestring=[b]",
  morestring=[b]',
  morestring=[b]"""
}
\lstset{
  language=scala,
  basicstyle=\ttfamily\scriptsize,
  keywordstyle=\bfseries\color{blue},
  commentstyle=\itshape\color{green!60!black},
  stringstyle=\color{orange},
  % numbers=left,
  % numberstyle=\tiny\color{gray},
  % stepnumber=1,
  % numbersep=5pt,
  backgroundcolor=\color{gray!5},
  frame=single,
  showspaces=false,
  showstringspaces=false,
  showtabs=false,
  tabsize=2,
  captionpos=b,
  breaklines=true,
  breakatwhitespace=false,
  escapeinside={\%*}{*)},
}

    
\begin{document}

\title{\huge \textsc{Informe Proyecto Final Programación Funcional y Concurrente - Problema de la Reconstrucción de ADN}\\
%{\footnotesize \textsuperscript{*}Note: Sub-titles are not captured in Xplore and
%should not be used}
%\thanks{Identify applicable funding agency here. If none, delete this.}
}

\author{\IEEEauthorblockN{Víctor Manuel Hernández Ortiz}
\IEEEauthorblockA{\textit{2259520} \\
\textit{Universidad Del Valle}\\}
\and
\IEEEauthorblockN{Jhon Alejandro Martinez Murillo}
\IEEEauthorblockA{\textit{2259565} \\
\textit{Universidad Del Valle}\\
}
}

\maketitle




\section{\textbf{Corrección de las funciones implementadas}}

\textbf{Notitas: *Particularmente debe describir su implementacion
de trie y por que son correctas las funciones pertenece, adicionar y arbolDeSufijos. No
olvide senalar tambien que colecciones paralelas fueron utilizadas.}

\subsection{\textbf{Solución Ingenua}}

\subsection{\textbf{Solución Ingenua Paralela}}

\subsection{\textbf{Solución Mejorada}}

\subsection{\textbf{Solución Mejorada Paralela}}

\subsection{\textbf{Turbo Solución}}

\subsection{\textbf{Turbo Solución Paralela}}

\subsection{\textbf{Turbo Mejorada}}

\subsection{\textbf{Turbo Mejorada Paralela}}

\subsection{\textbf{Turbo Acelerada}}

\subsection{\textbf{Turbo Acelerada Paralela}}



\section{\textbf{Desempeño  y comparación de las soluciones secuenciales y paralelas} }

\textbf{Notita: Describir como se hicieron las pruebas de desempeño de los algoritmos}


\subsection{\textbf{Desempeño de todas las soluciones juntas}}

Estas tres tablas representan una en realidad, pero por efectos visuales decidimos anexarla de esta manera

\begin{center}
\renewcommand{\arraystretch}{1}
\begin{tabularx}{\linewidth}{>{\centering\arraybackslash}X | >{\centering\arraybackslash}X | >{\centering\arraybackslash}X |}
    \toprule
    \textbf{Tamaño de la matriz} & \textbf{Multiplicación secuencial} & \textbf{Multiplicación paralela}  \\
    \midrule
    2 & 0.023874999 & 0.08563700 \\
    4 & 0.050001 & 0.084999 \\
    8 & 0.122071 &  0.0621710 \\
    16 & 0.091161 & 0.1137009 \\
    32 & 0.611 & 0.3503369 \\
    64 & 6.267858 & 3.0618850 \\
    128 & 79.0450319 & 39.5588989 \\
    \bottomrule
\end{tabularx}

\vspace{0.2cm}
\begin{tabularx}{\linewidth}{>{\centering\arraybackslash}X | >{\centering\arraybackslash}X | >{\centering\arraybackslash}X | }
    \toprule
    \textbf{Tamaño de la matriz} & \textbf{Multiplicación recursiva} & \textbf{Multiplicación recursiva paralela}  \\
    \midrule
    2 & 0.068845999 & 0.19054500 \\
    4 & 0.050001 & 0.084999 \\
    8 & 0.3908499 & 2.1938079 \\
    16 & 2.5607809 & 10.3008589 \\
    32 & 21.95624 & 54.2429610 \\
    64 & 203.3763620 & 369.499269 \\
    128 & 2491.38580 & 4278.7803739 \\
    \bottomrule
\end{tabularx}

\vspace{0.2cm}
\begin{tabularx}{\linewidth}{>{\centering\arraybackslash}X | >{\centering\arraybackslash}X | >{\centering\arraybackslash}X | }
    \toprule
    \textbf{Tamaño de la matriz} & \textbf{Algoritmo Strassen} & \textbf{Algoritmo Strassen paralelo} \\
    \midrule
    2 & 0.02743 & 0.074569 \\
    4 & 0.050001 & 0.084999 \\
    8 & 0.26881499 & 0.1974020\\
    16 & 2.0453940 &  1.0148579 \\
    32 & 14.9977109 & 6.1366830 \\
    64 & 129.294209 & 43.4983070 \\
    128 & 1417.1652550 & 400.7371680 \\
    \bottomrule
\end{tabularx}

\end{center}

De esta manera se calcularon los desempeños
\begin{lstlisting}
 //codigo de calculo de desempenos del benchmark
\end{lstlisting}





\subsection{\textbf{Desempeño de las soluciones secuenciales}}

\begin{table}[h]
    \centering
    \renewcommand{\arraystretch}{1.2}
    \begin{tabularx}{\linewidth}{>{\centering\arraybackslash}X | >{\centering\arraybackslash}X | >{\centering\arraybackslash}X | >{\centering\arraybackslash}X |}
        \toprule
        \textbf{Tamaño de la matriz} & \textbf{Multiplicación secuencial} & \textbf{Multiplicación recursiva} & \textbf{Strassen} \\
        \midrule
        2   & 0.0381420016 & 0.093908998 & 0.03746996 \\
        4   & 0.0247579992 & 0.109804999 & 0.129887006 \\
        8   & 0.0386939998 & 0.486354006 & 0.215725003 \\
        16  & 0.0933949998 & 2.657260997 & 1.437317997 \\
        32  & 0.4390439993 & 21.21956999 & 9.674953 \\
        64  & 5.3342230015 & 177.01692393 & 67.98518801 \\
        128 & 43.635844 & 1463.531014998 & 506.11463989 \\
        \bottomrule
    \end{tabularx}
\end{table}

\subsection{\textbf{Desempeño de las soluciones paralelas}}
\begin{table}[h]
    \centering
    \renewcommand{\arraystretch}{1.2}
    \begin{tabularx}{\linewidth}{>{\centering\arraybackslash}X | >{\centering\arraybackslash}X | >{\centering\arraybackslash}X | >{\centering\arraybackslash}X |}
        \toprule
        \textbf{Tamaño de la matriz} & \textbf{Multiplicación secuencial paralela} & \textbf{Multiplicación recursiva paralela} & \textbf{Strassen pararelo} \\
        \midrule
        2   & 0.0490440004 & 0.1274450003 & 0.08126299 \\
        4   & 0.0294580005 & 0.2960579998 & 0.11990699 \\
        8   & 0.0434640001 & 0.975459004 & 0.234464 \\
        16  & 0.0934720003 & 5.341355 & 1.2639369999 \\
        32  & 0.3401179999 & 31.503583 & 9.300731003 \\
        64  & 3.744225994 & 230.046051001 & 54.591446995 \\
        128 & 33.363751 & 1723.101270994 & 429.24941 \\
        \bottomrule
    \end{tabularx}
\end{table}

\newpage
De esta manera se calcularon los desempeños para las algoritmos secuenciales y paralelos:
\begin{lstlisting}
// codigo del calculo de algoritmos secuenciales y paralelos
\end{lstlisting}

\section{\textbf{Comparación de las distintas soluciones y sus paralelas}}

\subsection{\textbf{Comparación de la solución ingenua y la ingenua Paralela}}
\begin{table}[h]
    \centering
    \renewcommand{\arraystretch}{1.2}
    \begin{tabularx}{\linewidth}{>{\centering\arraybackslash}X | >{\centering\arraybackslash}X | >{\centering\arraybackslash}X | >{\centering\arraybackslash}X |}
        \toprule
        \textbf{Tamaño de la matriz} & \textbf{Multiplicación secuencial} & \textbf{Multiplicación paralela} & \textbf{Aceleracion} \\
        \midrule
        2   & 0.1941 & 0.3109 & 0.624316500482 \\
        4   & 0.0969 & 0.1253 & 0.773343974461 \\
        8   & 0.1558 & 0.512 & 0.304296874997 \\
        16  & 16.8657 & 0.4238 & 39.7963662104 \\
        32  & 1.0945 & 0.8826 & 1.24008610922 \\
        64  & 9.3597 & 2.6625 & 3.51538028169 \\
        128 & 74.3937 & 23.9784 & 3.1025297767 \\
        256 & 582.3829 & 198.7335 & 2.93047171 \\
        512 & 4811.1526 & 1559.8417 & 3.0843851 \\
        1024 & 41274.5973 & 15009.3957 & 2.7499173 \\
        \bottomrule
    \end{tabularx}
\end{table}


\subsection{\textbf{Comparación de la solución Mejorada y la Mejorada Paralela}}
\begin{table}[h]
    \centering
    \renewcommand{\arraystretch}{1.2}
    \begin{tabularx}{\linewidth}{>{\centering\arraybackslash}X | >{\centering\arraybackslash}X | >{\centering\arraybackslash}X | >{\centering\arraybackslash}X |}
        \toprule
        \textbf{Tamaño de la matriz} & \textbf{recursiva} & \textbf{recursiva paralela} & \textbf{Aceleracion} \\
        \midrule
        2   & 0.4017 & 1.3084 & 0.3070162029 \\
        4   & 0.3693 & 1.4949 & 0.2470399357 \\
        8   & 1.4204 & 6.1935 & 0.2293372083 \\
        16  & 3.5153 & 18.4654 & 0.1903722638 \\
        32  & 128.1462 & 81.0312 & 1.5814427035 \\
        64  & 289.9554 & 532.3358 & 0.5446851404 \\
        128 & 2244.0366 & 2965.9211 & 0.756606977 \\
        256 & 18234.863 & 21556.7414 & 0.845900716 \\
        512 & 91069.015 & 110813.2341 & 0.821824358 \\
        \bottomrule
    \end{tabularx}
\end{table}

Esta comparación la hicimos solo hasta un tamaño 512, porque despues de \textbf{17h 10m 41s} de estar haciendo la comparación, nos volvio a lanzar el error mencionado anteriormente \textbf{out of memory}


\subsection{\textbf{Comparación de la solución Turbo con la Turbo Paralela}}
\begin{table}[h]
    \centering
    \renewcommand{\arraystretch}{1.2}
    \begin{tabularx}{\linewidth}{>{\centering\arraybackslash}X | >{\centering\arraybackslash}X | >{\centering\arraybackslash}X | >{\centering\arraybackslash}X |}
        \toprule
        \textbf{Tamaño de la matriz} & \textbf{Strassen} & \textbf{Strassen paralelo} & \textbf{Aceleracion} \\
        \midrule
        2   & 0.7311 & 0.6296 & 1.16121346886 \\
        4   & 0.7454 & 0.3804 & 1.95951629863 \\
        8   & 1.4337 & 0.8821 & 1.62532592676 \\
        16  & 9.8697 & 2.7326 & 3.61183488252 \\
        32  & 27.3466 & 6.5672 & 4.16411865026 \\
        64  & 149.8199 & 38.782 & 3.86312980248 \\
        128 & 1187.6159 & 339.3482 & 3.4996970663 \\
        256 & 8264.3622 & 2360.687 & 3.5008292924 \\
        512 & 54366.4652 & 18073.1638 & 3.00813215 \\
        1024 & 399360.8654 & 132219.4327 & 3.020440 \\
        \bottomrule
    \end{tabularx}
\end{table}

De esta manera se calcularon las comparaciones
\begin{lstlisting}
// codigo
\end{lstlisting}

\subsection{\textbf{Comparación de la solución Turbo Mejorada con la Turbo Mejorada Paralela}}
\begin{table}[h]
    \centering
    \renewcommand{\arraystretch}{1.2}
    \begin{tabularx}{\linewidth}{>{\centering\arraybackslash}X | >{\centering\arraybackslash}X | >{\centering\arraybackslash}X | >{\centering\arraybackslash}X |}
        \toprule
        \textbf{Tamaño de la matriz} & \textbf{Producto punto secuencial} & \textbf{Producto punto paralelo} & \textbf{Aceleracion} \\
        \midrule
        $10^{1}$ & 0.0725 & 2.6356 & 0.02750796782\\
        $10^{2}$ & 0.0677 & 1.3331 & 0.05078388718 \\
        $10^{3}$ & 0.2188 & 1.2477 & 0.175362667307 \\
        $10^{4}$ & 2.197 & 4.4658 & 0.49196112678 \\
        $10^{5}$ & 8.7597 & 5.8216 & 1.50468943245 \\
        $10^{6}$ & 27.6552 & 40.2081 & 0.68780171159 \\
        $10^{7}$ & 338.2786 & 548.42 & 0.61682396703 \\
        \bottomrule
    \end{tabularx}
\end{table}

De esta manera se calcularon las comparaciones
\begin{lstlisting}
// codigo
\end{lstlisting}


\subsection{\textbf{Comparación de la solución Turbo Acelerada con la Turbo Acelerada Paralela}}


\section{\textbf{Análisis comparativo de las diferentes soluciones y conclusiones}}

\textbf{Analisis comparativo de multiplicacion secuencial vs multiplicacion paralelo}



\textbf{Analisis comparativo de multiplicacion recursiva vs multiplicacion recursiva paralelo}


\textbf{Analisis comparativo de strassen secuencial vs strassen paralelo}



\textbf{Analisis comparativo de algoritmos secuenciales}



\textbf{Analisis comparativo de algoritmos paralelos}



\textbf{Analisis comparativo de producto punto secuencial vs producto punto paralelo}




\textbf{¿Las paralelizaciones sirvieron?}



\textbf{¿Es realmente mas eficiente el algoritmo de Strassen? }


\textbf{¿No se puede concluir nada al respecto?} 


% \begin{center}
%     \begin{tabular}{| c | c | c | c | c | c | c | }
%     \hline Secuencial & Paralela & Recursiva & Recursiva Paralela & Strassen & Strassen Paralela  \\ \hline
%     España & Madrid & 2.000.000\\ \hline
%     Francia & París & 2.000.000\\ \hline
%     Alemania & Berlín & 2.000.000 \\ \hline
%     \end{tabular}
%     \end{center}
\section{\textbf{Pruebas de software}}

\href{https://github.com/JAMM0118/taller4-pfc-2023-2/tree/Alejandro}
{Link del codigo Fuente} \\

Las pruebas de software que realizamos fueron las siguientes:

\subsection{Comprobación que los algoritmos multiplicaran las matrices de manera correcta}
\subsection{Comparación de cada algoritmo secuencial con su version paralela}
\subsection{Comparación de los productos punto secuencial y paralelo}
\subsection{Evaluación de desempeño de todos los algoritmos}
\subsection{Evaluación de desempeño de todos los algoritmos secuenciales}
\subsection{Evaluación de desempeño de todos los algoritmos paralelos}
\subsection{Evaluación de desempeño de los productos punto secuencial y paralelo}


 
\end{document}