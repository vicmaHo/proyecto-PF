\documentclass[conference]{IEEEtran}
% \IEEEoverridecommandlockouts
% The preceding line is only needed to identify funding in the first footnote. If that is unneeded, please comment it out.
\usepackage{tabularx}
\usepackage{booktabs}
\usepackage{cite}
\usepackage{amsmath,amssymb,amsfonts}
\usepackage{algorithmic}
\usepackage{graphicx}
\usepackage{textcomp}
\usepackage{xcolor}
\def\BibTeX{{\rm B\kern-.05em{\sc i\kern-.025em b}\kern-.08em
    T\kern-.1667em\lower.7ex\hbox{E}\kern-.125emX}}
\usepackage{listings}
\usepackage{color}
\usepackage{booktabs}  % Para líneas de alta calidad en las tablas
\definecolor{dkgreen}{rgb}{0,0.6,0}
\definecolor{gray}{rgb}{0.5,0.5,0.5}
\definecolor{mauve}{rgb}{0.58,0,0.82}
\usepackage{hyperref}
\hypersetup{
    colorlinks=true,    % Activa la coloración de enlaces
    linkcolor=black,    % Color de enlaces internos (por ejemplo, índice)
    urlcolor=blue,      % Color de enlaces a URLs
    citecolor=red       % Color de enlaces de citas bibliográficas
}

% \usepackage{geometry}
% \usepackage{graphicx}
% \geometry{letterpaper, top = 2.5cm, right = 2.5cm, left = 2.5cm, bottom = 2.5cm}


\lstdefinelanguage{scala}{
  morekeywords={abstract,case,catch,class,def,%
    else,extends,false,final,finally,%
    for,if,implicit,import,match,mixin,%
    new,null,object,override,package,%
    private,protected,requires,return,sealed,%
    super,this,throw,trait,true,try,%
    type,val,var,while,with,yield},
  otherkeywords={=>,<-,<\%,<:,>:,\#,@},
  sensitive=true,
  morecomment=[l]{//},
  morecomment=[n]{/*}{*/},
  morestring=[b]",
  morestring=[b]',
  morestring=[b]"""
}
\lstset{
  language=scala,
  basicstyle=\ttfamily\scriptsize,
  keywordstyle=\bfseries\color{blue},
  commentstyle=\itshape\color{green!60!black},
  stringstyle=\color{orange},
  % numbers=left,
  % numberstyle=\tiny\color{gray},
  % stepnumber=1,
  % numbersep=5pt,
  backgroundcolor=\color{gray!5},
  frame=single,
  showspaces=false,
  showstringspaces=false,
  showtabs=false,
  tabsize=2,
  captionpos=b,
  breaklines=true,
  breakatwhitespace=false,
  escapeinside={\%*}{*)},
}

    
\begin{document}

\title{\huge \textsc{Informe Proyecto Final Programación Funcional y Concurrente - Problema de la Reconstrucción de ADN}\\
%{\footnotesize \textsuperscript{*}Note: Sub-titles are not captured in Xplore and
%should not be used}
%\thanks{Identify applicable funding agency here. If none, delete this.}
}

\author{\IEEEauthorblockN{Víctor Manuel Hernández Ortiz}
\IEEEauthorblockA{\textit{2259520} \\
\textit{Universidad Del Valle}\\}
\and
\IEEEauthorblockN{Jhon Alejandro Martinez Murillo}
\IEEEauthorblockA{\textit{2259565} \\
\textit{Universidad Del Valle}\\
}
}

\maketitle




\section{\textbf{Corrección de las funciones implementadas}}

\textbf{Notitas: *Particularmente debe describir su implementacion
de trie y por que son correctas las funciones pertenece, adicionar y arbolDeSufijos. No
olvide senalar tambien que colecciones paralelas fueron utilizadas.}

\subsection{\textbf{Solución Ingenua}}

Para la solución ingenua, se genera una lista de combinaciones posibles para cadenas del tamaño de la que se desea hallar, posteriormente se recorre cada cadena de las combinaciones posibles y se le pregunta al $oraculo$ por esta, continua hasta finalizar y por ultimo se retorna el resultado.

\begin{lstlisting}
def reconstruirCadenaIngenuo(n: Int, oraculo: Oraculo): Seq[Char] = {
    val combinaciones = generarCombinaciones(n)
    val cadenaEncontrada = for {
        cadena <- combinaciones
        if oraculo(cadena) == true
    } yield cadena.toSeq
    
    cadenaEncontrada.head
}
\end{lstlisting}

\subsection{\textbf{Solución Ingenua Paralela}}

Para la solución ingenua paralela tuvimos dos enfoques, paralelización de datos y paralelización de tareas. Inicialmente notamos que la instrucción $for$ (que hace el recorrido analizando con el $oraculo$ cual de las cadenas pertenecientes a la lista de todas las combinaciones posibles es la correcta) podía ser dividida en varias partes. Si, por ejemplo, tenemos un tamaño de cadena $n=4$ eso nos arrojaría una lista de combinaciones $4^n$, para este ejemplo $ 4^4 = 256$ sería el tamaño de la lista, entonces podemos dividir esta lista de combinaciones en 4 partes, cada una de tamaño $64$, y con ello usar un $for$ para cada sub-lista de tamaño $64$ de forma paralela, de esta forma cada instrucción $for$ se ejecuta simultáneamente a las otras, lo que en teoría se traduce a menos tiempo para hallar la cadena correcta. Otro añadido es que se cambia la instrucción $for$ por un $filter$ que realiza la misma acción, hallar la cadena mediante el $oraculo$, pero gracias al $filter$ podemos usar paralelización de datos a la secuencia que se filtrara. De esta forma se usa una secuencia de datos paralela, aplicando paralelismo de datos, y cada comparación de las sub-listas de combinaciones se ejecuta de forma paralela.

\begin{lstlisting}
//tarea de recorrido, recibe una porcion de la lista de combinaciones
def tareaRecorrido(bloqueCombinaciones: Seq[String]) : Seq[Char] = {
    val combinacionesFiltradas = bloqueCombinaciones.par.filter(oraculo(_) == true)
    if (combinacionesFiltradas.isEmpty) {
        Seq.empty[Char] 
    } else {
        val combinacion = combinacionesFiltradas.head
        combinacion.toSeq
    }  
}
// separo en bloques
val (bloque1, bloque2, bloque3, bloque4) = separarCombinaciones(combinaciones)

// Ejecuto las tareas sobre cada bloque
val (resultado1, resultado2, resultado3, resultado4) = parallel(tareaRecorrido(bloque1), tareaRecorrido(bloque2), tareaRecorrido(bloque3), tareaRecorrido(bloque4))
\end{lstlisting}
Finalmente elijo el resultado que no se encuentre vacío, ya que solo una $tareaRecorrido$ arrojará la cadena que se trata de reconstruir.

\subsection{\textbf{Solución Mejorada}}

\subsection{\textbf{Solución Mejorada Paralela}}

\textbf{Describo el funcionamiento que se implementa anterioremente}
El añadido para la paralelización, consta de usar el mismo concepto de dividir el $for$ en varias tareas paralelas

\subsection{\textbf{Turbo Solución}}

\subsection{\textbf{Turbo Solución Paralela}}

\subsection{\textbf{Turbo Mejorada}}

\subsection{\textbf{Turbo Mejorada Paralela}}

\subsection{\textbf{Turbo Acelerada}}

\subsection{\textbf{Turbo Acelerada Paralela}}



\section{\textbf{Desempeño  y comparación de las soluciones secuenciales y paralelas} }

\textbf{Notita: Describir como se hicieron las pruebas de desempeño de los algoritmos}
Para las pruebas de desempeño, usamos la función $desempenoFunciones$, la cual prueba cada algoritmo 100 veces con cadenas distintas y para distintos tamaños, al final se promedian los resultados de tiempo
arrojando el resultado que se verá a continuación.

\subsection{\textbf{Desempeño de todas las soluciones juntas}}

Estas tres tablas representan una en realidad, pero por efectos visuales decidimos anexarla de esta manera

% \begin{center}
% \renewcommand{\arraystretch}{1}
% \begin{tabularx}{\linewidth}{>{\centering\arraybackslash}X | >{\centering\arraybackslash}X | >{\centering\arraybackslash}X |}
%     \toprule
%     \textbf{Tamaño de la matriz} & \textbf{Multiplicación secuencial} & \textbf{Multiplicación paralela}  \\
%     \midrule
%     2 & 0.023874999 & 0.08563700 \\
%     4 & 0.050001 & 0.084999 \\
%     8 & 0.122071 &  0.0621710 \\
%     16 & 0.091161 & 0.1137009 \\
%     32 & 0.611 & 0.3503369 \\
%     64 & 6.267858 & 3.0618850 \\
%     128 & 79.0450319 & 39.5588989 \\
%     \bottomrule
% \end{tabularx}


\begin{table}[h]
    \centering
    \renewcommand{\arraystretch}{1.2}
    \begin{tabularx}{\linewidth}{>{\centering\arraybackslash}X | >{\centering\arraybackslash}X | >{\centering\arraybackslash}X | >{\centering\arraybackslash}X |}
        \toprule
        \textbf{Tamaño} & \textbf{Ingenua} & \textbf{Ingenua Paralela} & \textbf{Mejorada} \\
        \midrule
        2   & 0.0381420016 & 0.093908998 & 0.03746996 \\
        3   & 0.0247579992 & 0.109804999 & 0.129887006 \\
        4   & 0.0386939998 & 0.486354006 & 0.215725003 \\
        5  & 0.0933949998 & 2.657260997 & 1.437317997 \\
        6  & 0.4390439993 & 21.21956999 & 9.674953 \\
        7  & 5.3342230015 & 177.01692393 & 67.98518801 \\
        8 & 43.635844 & 1463.531014998 & 506.11463989 \\
        9 & 43.635844 & 1463.531014998 & 506.11463989 \\
        10 & 43.635844 & 1463.531014998 & 506.11463989 \\
        \bottomrule
    \end{tabularx}
\end{table}

\vspace{0.2cm}

\begin{table}[h]
    \centering
    \renewcommand{\arraystretch}{1.2}
    \begin{tabularx}{\linewidth}{>{\centering\arraybackslash}X | >{\centering\arraybackslash}X | >{\centering\arraybackslash}X | >{\centering\arraybackslash}X |}
        \toprule
        \textbf{Tamaño} & \textbf{Mejorada Paralela} & \textbf{Turbo Solución} & \textbf{Turbo Solución Paralela} \\
        \midrule
        2   & 0.0381420016 & 0.093908998 & 0.03746996 \\
        3   & 0.0247579992 & 0.109804999 & 0.129887006 \\
        4   & 0.0386939998 & 0.486354006 & 0.215725003 \\
        5  & 0.0933949998 & 2.657260997 & 1.437317997 \\
        6  & 0.4390439993 & 21.21956999 & 9.674953 \\
        7  & 5.3342230015 & 177.01692393 & 67.98518801 \\
        8 & 43.635844 & 1463.531014998 & 506.11463989 \\
        9 & 43.635844 & 1463.531014998 & 506.11463989 \\
        10 & 43.635844 & 1463.531014998 & 506.11463989 \\
        \bottomrule
    \end{tabularx}
\end{table}

\vspace{0.2cm}


\begin{table}[h]
    \centering
    \renewcommand{\arraystretch}{1.2}
    \begin{tabularx}{\linewidth}{>{\centering\arraybackslash}X | >{\centering\arraybackslash}X | >{\centering\arraybackslash}X | >{\centering\arraybackslash}X |}
        \toprule
        \textbf{Tamaño} & \textbf{Turbo Mejorada} & \textbf{Turbo Mejorada Paralela} & \textbf{Turbo Acelerada} \\
        \midrule
        2   & 0.0381420016 & 0.093908998 & 0.03746996 \\
        3   & 0.0247579992 & 0.109804999 & 0.129887006 \\
        4   & 0.0386939998 & 0.486354006 & 0.215725003 \\
        5  & 0.0933949998 & 2.657260997 & 1.437317997 \\
        6  & 0.4390439993 & 21.21956999 & 9.674953 \\
        7  & 5.3342230015 & 177.01692393 & 67.98518801 \\
        8 & 43.635844 & 1463.531014998 & 506.11463989 \\
        9 & 43.635844 & 1463.531014998 & 506.11463989 \\
        10 & 43.635844 & 1463.531014998 & 506.11463989 \\
        \bottomrule
    \end{tabularx}
\end{table}

\newpage

\begin{table}[h]
    \centering
    \renewcommand{\arraystretch}{1.2}
    \begin{tabularx}{\linewidth}{>{\centering\arraybackslash}X | >{\centering\arraybackslash}X |}
        \toprule
        \textbf{Tamaño} & \textbf{Turbo Acelerada Paralela}  \\
        \midrule
        2   & 0.0381420016 \\
        3   & 0.0247579992\\
        4   & 0.0386939998 \\
        5  & 0.0933949998 \\
        6  & 0.4390439993 \\
        7  & 5.3342230015 \\
        8 & 43.635844 \\
        9 & 43.635844 \\
        10 & 43.635844 \\
        \bottomrule
    \end{tabularx}
\end{table}

% \begin{tabularx}{\linewidth}{>{\centering\arraybackslash}X | >{\centering\arraybackslash}X |}
%     \toprule
%     \textbf{Tamaño} & \textbf{Turbo Acelerada Paralela} & \textbf{Algoritmo Strassen paralelo} \\
%     \midrule
%     2 & 0.02743 \\
%     3 & 0.050001\\
%     4 & 0.26881499\\
%     5 & 2.0453940 \\
%     6 & 14.9977109  \\
%     7 & 129.294209  \\
%     8 & 1417.1652550 \\
%     9 & 1417.1652550 \\
%     10 & 1417.1652550 \\
%     \bottomrule
% \end{tabularx}\\



De esta manera se calcularon los desempeños
\begin{lstlisting}
// Primera parte de la funcion
 def desempenoDeFunciones(tamanoCadena: Int): Vector[Double] = {
    println("Tamanio: " + tamanoCadena)

    // repido 100 veces una reconstruccion para una cadena distinta y al final promedio los resultados, esto se repite para todas las funciones
    val tiemposIngenua = (1 to 100).map(_ => 0.0).toArray
    for (i <- 0 until 100) {
    val time = withWarmer(new Warmer.Default) measure {
        val cadenaAleatoria = crearADN(tamanoCadena)
        val oraculo = oraculoFunc(cadenaAleatoria)
        reconstruirCadenaIngenuo(cadenaAleatoria.length, oraculo)
    }
    tiemposIngenua(i) = time.value
    }
    ...
    ...

    // resultado, promedios de todas las funciones
     Vector(tiemposIngenua.sum / 100, tiemposIngenuaPar.sum / 100, tiempoMejorado.sum / 100, tiempoMejoradoPar.sum / 100, tiempoTurbo.sum / 100, tiempoTurboPar.sum / 100, tiempoTurboMejorada.sum / 100, tiempoTurboMejoradaPar.sum / 100, tiempoTurboAcelerada.sum / 100, tiempoTurboAceleradaPar.sum / 100)
}
\end{lstlisting}





\subsection{\textbf{Desempeño de las soluciones secuenciales}}

\begin{table}[h]
    \centering
    \renewcommand{\arraystretch}{1.2}
    \begin{tabularx}{\linewidth}{>{\centering\arraybackslash}X | >{\centering\arraybackslash}X | >{\centering\arraybackslash}X | >{\centering\arraybackslash}X |>{\centering\arraybackslash}X |>{\centering\arraybackslash}X |}
        \toprule
        \textbf{Tamaño} & \textbf{Ingenua} & \textbf{Mejorada} & \textbf{Turbo} & \textbf{Turbo Mejorada} & \textbf{Turbo Acelerada} \\
        \midrule
        2   & 0.0381 & 0.09390 & 0.03746& 0.09390& 0.0374 \\
        3   & 0.02475 & 0.10980 & 0.12988 & 0.09390 & 0.0374 \\
        4   & 0.03869 & 0.48635 & 0.2157  & 0.09390 & 0.0374\\
        5  & 0.0933 & 2.65726 & 1.4373 & 0.09390 & 0.0374\\
        6  & 0.43904& 21.2195 & 9.6749 & 0.09390 & 0.03746\\
        7  & 5.33422 & 177.0169 & 67.9851  & 0.09390 & 0.03746\\
        8 & 43.635& 1463.5310 & 506.1146 & 0.09390 & 0.03746\\
        9 & 43.635& 1463.5310 & 506.1146 & 0.09390 & 0.03746\\
        10 & 43.635& 1463.5310 & 506.1146 & 0.09390 & 0.03746\\
        \bottomrule
    \end{tabularx}
\end{table}

\subsection{\textbf{Desempeño de las soluciones paralelas}}
\begin{table}[h]
    \centering
    \renewcommand{\arraystretch}{1.2}
    \begin{tabularx}{\linewidth}{>{\centering\arraybackslash}X | >{\centering\arraybackslash}X | >{\centering\arraybackslash}X | >{\centering\arraybackslash}X |>{\centering\arraybackslash}X |>{\centering\arraybackslash}X |}
        \toprule
        \textbf{Tamaño} & \textbf{Ingenua Paralela} & \textbf{Mejorada Paralela} & \textbf{Turbo Paralela} & \textbf{Turbo Mejorada Paralela} & \textbf{Turbo Acelerada Paralela} \\
        \midrule
        2   & 0.0381 & 0.09390 & 0.03746& 0.09390& 0.0374 \\
        3   & 0.02475 & 0.10980 & 0.12988 & 0.09390 & 0.0374 \\
        4   & 0.03869 & 0.48635 & 0.2157  & 0.09390 & 0.0374\\
        5  & 0.0933 & 2.65726 & 1.4373 & 0.09390 & 0.0374\\
        6  & 0.43904& 21.2195 & 9.6749 & 0.09390 & 0.03746\\
        7  & 5.33422 & 177.0169 & 67.9851  & 0.09390 & 0.03746\\
        8 & 43.635& 1463.5310 & 506.1146 & 0.09390 & 0.03746\\
        9 & 43.635& 1463.5310 & 506.1146 & 0.09390 & 0.03746\\
        10 & 43.635& 1463.5310 & 506.1146 & 0.09390 & 0.03746\\
        \bottomrule
    \end{tabularx}
\end{table}


\newpage
De esta manera se calcularon los desempeños para las algoritmos secuenciales y paralelos:
\begin{lstlisting}
// codigo del calculo de algoritmos secuenciales y paralelos
\end{lstlisting}

\section{\textbf{Comparación de las distintas soluciones y sus paralelas}}

\subsection{\textbf{Comparación de la solución ingenua y la ingenua Paralela}}
\begin{table}[h]
    \centering
    \renewcommand{\arraystretch}{1.2}
    \begin{tabularx}{\linewidth}{>{\centering\arraybackslash}X | >{\centering\arraybackslash}X | >{\centering\arraybackslash}X | >{\centering\arraybackslash}X |}
        \toprule
        \textbf{Tamaño} & \textbf{Ingenua} & \textbf{Ingenua Paralela} & \textbf{Aceleración} \\
        \midrule
        2   & 0.1941 & 0.3109 & 0.624316500482 \\
        3   & 0.0969 & 0.1253 & 0.773343974461 \\
        4   & 0.1558 & 0.512 & 0.304296874997 \\
        5  & 16.8657 & 0.4238 & 39.7963662104 \\
        6  & 1.0945 & 0.8826 & 1.24008610922 \\
        7  & 9.3597 & 2.6625 & 3.51538028169 \\
        8 & 74.3937 & 23.9784 & 3.1025297767 \\
        9 & 582.3829 & 198.7335 & 2.93047171 \\
        10 & 4811.1526 & 1559.8417 & 3.0843851 \\
        \bottomrule
    \end{tabularx}
\end{table}


\subsection{\textbf{Comparación de la solución Mejorada y la Mejorada Paralela}}
\begin{table}[h]
    \centering
    \renewcommand{\arraystretch}{1.2}
    \begin{tabularx}{\linewidth}{>{\centering\arraybackslash}X | >{\centering\arraybackslash}X | >{\centering\arraybackslash}X | >{\centering\arraybackslash}X |}
        \toprule
        \textbf{Tamaño} & \textbf{Mejorada} & \textbf{Mejorada Paralela} & \textbf{Aceleración} \\
        \midrule
        2   & 0.4017 & 1.3084 & 0.3070162029 \\
        3   & 0.3693 & 1.4949 & 0.2470399357 \\
        4   & 1.4204 & 6.1935 & 0.2293372083 \\
        5  & 3.5153 & 18.4654 & 0.1903722638 \\
        6  & 128.1462 & 81.0312 & 1.5814427035 \\
        7  & 289.9554 & 532.3358 & 0.5446851404 \\
        8 & 2244.0366 & 2965.9211 & 0.756606977 \\
        9 & 18234.863 & 21556.7414 & 0.845900716 \\
        10 & 91069.015 & 110813.2341 & 0.821824358 \\
        \bottomrule
    \end{tabularx}
\end{table}


\newpage

\subsection{\textbf{Comparación de la solución Turbo con la Turbo Paralela}}
\begin{table}[h]
    \centering
    \renewcommand{\arraystretch}{1.2}
    \begin{tabularx}{\linewidth}{>{\centering\arraybackslash}X | >{\centering\arraybackslash}X | >{\centering\arraybackslash}X | >{\centering\arraybackslash}X |}
        \toprule
        \textbf{Tamaño} & \textbf{Turbo} & \textbf{Turbo Paralela} & \textbf{Aceleración} \\
        \midrule
        2   & 0.7311 & 0.6296 & 1.16121346886 \\
        3   & 0.7454 & 0.3804 & 1.95951629863 \\
        4   & 1.4337 & 0.8821 & 1.62532592676 \\
        5  & 9.8697 & 2.7326 & 3.61183488252 \\
        6  & 27.3466 & 6.5672 & 4.16411865026 \\
        7  & 149.8199 & 38.782 & 3.86312980248 \\
        8 & 1187.6159 & 339.3482 & 3.4996970663 \\
        9 & 8264.3622 & 2360.687 & 3.5008292924 \\
        10 & 54366.4652 & 18073.1638 & 3.00813215 \\
        \bottomrule
    \end{tabularx}
\end{table}


\subsection{\textbf{Comparación de la solución Turbo Mejorada con la Turbo Mejorada Paralela}}
\begin{table}[h]
    \centering
    \renewcommand{\arraystretch}{1.2}
    \begin{tabularx}{\linewidth}{>{\centering\arraybackslash}X | >{\centering\arraybackslash}X | >{\centering\arraybackslash}X | >{\centering\arraybackslash}X |}
        \toprule
        \textbf{Tamaño} & \textbf{Turbo Mejorada} & \textbf{Turbo Mejorada Paralela} & \textbf{Aceleración} \\
        \midrule
        2   & 0.7311 & 0.6296 & 1.16121346886 \\
        3   & 0.7454 & 0.3804 & 1.95951629863 \\
        4   & 1.4337 & 0.8821 & 1.62532592676 \\
        5  & 9.8697 & 2.7326 & 3.61183488252 \\
        6  & 27.3466 & 6.5672 & 4.16411865026 \\
        7  & 149.8199 & 38.782 & 3.86312980248 \\
        8 & 1187.6159 & 339.3482 & 3.4996970663 \\
        9 & 8264.3622 & 2360.687 & 3.5008292924 \\
        10 & 54366.4652 & 18073.1638 & 3.00813215 \\
        \bottomrule
    \end{tabularx}
\end{table}






\subsection{\textbf{Comparación de la solución Turbo Acelerada con la Turbo Acelerada Paralela}}
\begin{table}[h]
    \centering
    \renewcommand{\arraystretch}{1.2}
    \begin{tabularx}{\linewidth}{>{\centering\arraybackslash}X | >{\centering\arraybackslash}X | >{\centering\arraybackslash}X | >{\centering\arraybackslash}X |}
        \toprule
        \textbf{Tamaño} & \textbf{Turbo Acelerada} & \textbf{Turbo Acelerada Paralela} & \textbf{Aceleración} \\
        \midrule
        2   & 0.7311 & 0.6296 & 1.16121346886 \\
        3   & 0.7454 & 0.3804 & 1.95951629863 \\
        4   & 1.4337 & 0.8821 & 1.62532592676 \\
        5  & 9.8697 & 2.7326 & 3.61183488252 \\
        6  & 27.3466 & 6.5672 & 4.16411865026 \\
        7  & 149.8199 & 38.782 & 3.86312980248 \\
        8 & 1187.6159 & 339.3482 & 3.4996970663 \\
        9 & 8264.3622 & 2360.687 & 3.5008292924 \\
        10 & 54366.4652 & 18073.1638 & 3.00813215 \\
        \bottomrule
    \end{tabularx}
\end{table}



De esta manera se calcularon las comparaciones
\begin{lstlisting}
// codigo
\end{lstlisting}

\section{\textbf{Análisis comparativo de las diferentes soluciones y conclusiones}}



% \begin{center}
%     \begin{tabular}{| c | c | c | c | c | c | c | }
%     \hline Secuencial & Paralela & Recursiva & Recursiva Paralela & Strassen & Strassen Paralela  \\ \hline
%     España & Madrid & 2.000.000\\ \hline
%     Francia & París & 2.000.000\\ \hline
%     Alemania & Berlín & 2.000.000 \\ \hline
%     \end{tabular}
%     \end{center}

 
\end{document}